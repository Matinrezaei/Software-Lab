\documentclass[a4paper]{article}
\input{etc/cmd}
\settextfont[Scale=1.2,BoldFont=B Titr]{B Zar}

\begin{document}

	{ \centering
		\begin{figure}[ht]
			\centering
			\includegraphics[width=0.5\linewidth]{etc/IUTLogo}
			\label{fig:fig1}
		\end{figure}
		
		\fontsize{28pt}{35pt}{\textbf{دانشگاه صنعتی اصفهان}}\\
		
		\vspace{1cm}
		
		\fontsize{28pt}{35pt}{\bf دانشکده برق و کامپیوتر}
		
		\vspace{1cm}
		
		\fontsize{25pt}{35pt}{\bf نیازمندی‌های پروژه مهندسی نرم‌افزار}
		
		\vspace{2cm}
		
	 \begin{huge}
	 	\textbf{اعضای گروه} \\
	 	\fontsize{20}{20} \vspace{0.5cm}
	 	\begin{center}
	 		پردیس یاوری -	پویا غنی - متین رضایی
	 	\end{center}
	 
	 \end{huge}
}
	
	\newpage
	\thispagestyle{empty}
	\pagenumbering{arabic}
	\tableofcontents
	
	\newpage

	\fontsize{12pt}{14pt}\selectfont
	
	\pagestyle{fancy}

	\setlength\headheight{26pt}
	
	\rhead{\begin{picture}(3,3) \put(3,3){\includegraphics[width=2cm]{etc/IUTLogo}} \end{picture}}
	
	\lhead{\begin{subject}
			نیازمندی‌های پروژه مهندسی نرم‌افزار
			\vspace{1ex plus 0.5ex minus 0.5ex}
	\end{subject}
	}
	
	\section{\lr{functional}}
	نیازمندی های \lr{functional}  نرم افزار شامل:\\
	هر کاربر بدون نقشی با وارد شدن به پلتفرم روان سبز می تواند از خدمات زیر بهره مند شود:
	\begin{itemize}
		
		\item مجله هفتگی، پادکست ها، تست های روان شناسی، مشاهده پزشکان و روان شناسان
		\item هر کاربر بدون نقش یا با نقش میتواند در مجله های هفتگی مطلب پیشنهادی خود را انتشار دهد و در تولید آن سهیم باشد.
		
	\end{itemize}
		
		\subsection{بیمار}
		
		\begin{itemize}
			
		\item کاربر با نقش بیمار میتواند با استفاده از گزینه ثبت نام پنل خود را ایجاد کند.
		\item کاربر با نقش بیمار میتواند با استفاده از گزینه ورود با وارد کردن نام کاربری و رمز عبور وارد پنل ساخته شده خود شود.
		\\
		کاربر با نقش بیمار بعد از ورود می تواند:
		\item کاربر با نقش بیمار می‌تواند با مراجعه به سایت از امکانات انواع مشاوره ها بهره مند شود. همچنین می‌تواند از امکان نوبت‌گیری استفاده نماید.
		\item هر کاربر برای استفاده از خدمات مشاوره، روان پزشکی، تهیه نسخه، دریافت نوبت باید به پنل خود ورود کند.
		\item کاربر فقط میتواند یک نوبت از هر پزشک بگیرد و تا زمانی که موعد نوبت سرنرسیده نمیتواند نوبت مجدد از آن پزشک تهیه کند.
		\item کاربر با نقش بیمار می‌تواند با استفاده از این پلتفرم، خدمات روان پزشکی، شامل مراجعه به روان‌پزشک و تهیه نسخه‌های درمانی از داروخانه استفاده کند.
		\item کاربر با نقش بیمار می‌تواند در زمان نوبت‌گیری، نوع مراجعه به پزشک را از بین، متنی، ویدیویی، صوتی یا حضوری انتخاب کند.
		\item هر کاربر با نقش بیمار امکان مشاهده مجله هفتگی روان سبز و همچنین استفاده از پادکست‌های مجموعه را خواهد داشت.
		\item همچنین کاربر بعد از ورود همچنان میتواند از امکانات مجله هفتگی، پادکست ها و تست های روانشناسی بهره مند شود.
		\item کاربر با نقش بیمار می‌تواند با انتخاب پروفایل روان‌شناسان و پزشکان امکان نمره‌دهی و ثبت نظرات خود و کسب اطلاعات درباره آن‌ها را داشته باشد. (البته ثبت نظر را کسانی میتوانند انجام دهند که نوبتی با پزشک گذرانده باشند)
		\item کاربر با نقش بیمار می‌تواند با انجام فیلترهای تعیین شده براساس معیارهای مختلف، گزینه‌های مناسب خود را پیدا کند.
		\item هر کاربر، امکان انجام تست‌های روان‌شناسی موجود در پلتفرم روان سبز را دارد.
		\item هر کاربر میتواند نسخه های خود را مشاهده کند.
		\item هر کاربر میتواند نسخه های خود را از داروخانه های دلخواه استعلام بگیرد.
		
		\end{itemize}
		
		\subsection{پزشک و روان شناس}
		
		\begin{itemize}
			
		\item پزشک نیز به مانند بیمار میتواند از گزینه های ثبت نام و ورود استفاده کند.   	 
		\item هر پزشک یا روان‌شناس بعد از توافق همکاری امکان ایجاد اکانت و تکمیل پروفایل خود را دارد. این پروفایل شامل ویدئو معرفی و رزومه میباشد.
		\item پزشک فقط امکان مشاهده وضعیت بیماران خود را دارد.
		\item پزشک میتواند برای بیمار خود نسخه را ثبت کند.
		
		\end{itemize}
	
		\subsection{بهیار}
		
		\begin{itemize}
			
		\item بهیار‌های سیستم می‌توانند با ورود با بخش پروفایل مخصوص خود امکان دسترسی‌های مناسب برای پیگیری بیماران را داشته باشند.
		\item هر کاربر با نقش بهیار، می‌تواند بیمارن پزشکانی که او بهیار آن‌هاست را ببیند.
		\item بهیار میتواند با توجه به برنامه ریخته شده توسط پزشک برای هر بیمار، با بیمار تماس گرفته و جویای وضعیت وی باشد.
		\item بهیار می‌تواند درخواست ها را حذف یا قبول کند.
		\item بهیار به صورت هفتگی باید در قسمت همان بیمار وضعیت چک کردن را بروز می‌کند تا پزشک نیز از وضعیت بیمار اطلاع پیدا کند.
				
		\end{itemize}
		
		\subsection{داروخانه}
		داروخانه میتواند استعلام نسخه را بگیرد و نتیجه(داروهای موجود) را برای بیمار بفرستد.
	\section{\lr{non-functional}}
	نیازمندی های \lr{non-functional} نرم افزار شامل:
	
	\begin{itemize}
						
		\item پلتفرم باید اقدامات امنیتی قوی برای محافظت از محرمانه بودن، یکپارچگی و در دسترس بودن داده های بیمار و کاربر داشته باشد. این اقدامات شامل ذخیره سازی امن، کانال های ارتباطی امن، احراز هویت کاربر و کنترل دسترسی است.\\
		به منظور کنترل دسترسی، پزشکان باید به اطلاعات بیماران خود دسترسی داشته باشند. اطلاعات بیماران نباید در دسترس بقیه نقش‌های سیستم باشد.
		
		\item پلتفرم ما باید دارای پروانه و مجوز قانونی باشد.
		
		\item پلتفرم باید یک رابط کاربر پسند داشته باشد که به راحتی قابل پیمایش و درک باشد. این باید کاربران با سطوح مختلف تخصص فنی را در خود جای دهد و دستورالعمل ها و راهنمایی های روشنی را ارائه دهد.
		
		\item پلتفرم باید قابل اعتماد و برای استفاده 24 ساعته در دسترس باشد. باید زمان خرابی را به حداقل برساند و خطاها را با ظرافت مدیریت کند و پیام های خطای مناسبی را به کاربران ارائه دهد.
		
		\item پلتفرم باید به صورت ماژولار و قابل نگهداری طراحی و پیاده سازی شود. این شامل نوشتن کدهای تمیز و مستند، استفاده از الگوهای طراحی مناسب و پیروی از بهترین شیوه های کدنویسی است.
		
		\item پلتفرم باید قابلیت اجرا بر روی پلتفرم ها و محیط های مختلف را داشته باشد و به کاربران اجازه دهد از دستگاه های مختلف مانند رایانه های رومیزی، لپ تاپ، تبلت و گوشی های هوشمند به آن دسترسی داشته باشند.
		
		\item پلتفرم باید به گونه ای طراحی شود که تست را تسهیل کند. این امر قابلیت اطمینان و کیفیت پلتفرم را تضمین می کند.
		
		\item پلتفرم باید مقیاس پذیر باشد تا بتواند رشد آینده را از نظر پایگاه کاربر و حجم داده ها تطبیق دهد. باید بتواند تقاضاهای فزاینده را بدون کاهش قابل توجه عملکرد مدیریت کند.
		
		\item پلتفرم باید بتواند با سیستم ها و فناوری های دیگر مانند داروخانه ها و شرکت های بیمه هماهنگ شود تا تبادل اطلاعات و گردش کار یکپارچه را تسهیل کند.
		
	\end{itemize}

\end{document}
