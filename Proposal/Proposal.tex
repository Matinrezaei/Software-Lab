\documentclass[a4paper]{article}
\input{etc/cmd}
\settextfont[Scale=1.2,BoldFont=B Titr]{B Zar}

\begin{document}

	{ \centering
		\begin{figure}[ht]
			\centering
			\includegraphics[width=0.5\linewidth]{etc/IUTLogo}
			\label{fig:fig1}
		\end{figure}
		
		\fontsize{28pt}{35pt}{\textbf{دانشگاه صنعتی اصفهان}}\\
		
		\vspace{1cm}
		
		\fontsize{28pt}{35pt}{\bf دانشکده برق و کامپیوتر}
		
		\vspace{1cm}
		
		\fontsize{25pt}{35pt}{\bf پروپوزال پروژه مهندسی نرم‌افزار}
		
		\vspace{2cm}
		
	 \begin{huge}
	 	\textbf{اعضای گروه} \\
	 	\fontsize{20}{20} \vspace{0.5cm}
	 	\begin{center}
	 		پردیس یاوری -	پویا غنی - متین رضایی
	 	\end{center}
	 
	 \end{huge}
}

%\title{
%	\includegraphics[width=0.5\linewidth]{etc/IUTLogo} \\
		%دانشکده مهندسی کامپیوتر \\
%	}
%	\author{PMP}
%	\date{مهر ۱۴۰۲}
	
	\newpage
	\thispagestyle{empty}
	\pagenumbering{arabic}
	\tableofcontents
	
	\newpage

	\fontsize{12pt}{14pt}\selectfont
	
	\pagestyle{fancy}

	\setlength\headheight{26pt}
	
	\rhead{\begin{picture}(3,3) \put(3,3){\includegraphics[width=2cm]{etc/IUTLogo}} \end{picture}}
	
	\lhead{\begin{subject}
			پروپوزال پروژه مهندسی نرم‌افزار
			\vspace{1ex plus 0.5ex minus 0.5ex}
	\end{subject}
	}

		\section{بخش اول: برتری پروژه}
		\subsection{اسم پروژه}
 روان سبز
		\subsection{توضیح کلی پروژه}
		پروژه ما پلتفرمی است که خدمات مربوط به روان و اموزش‌های جامع مربوط به آن را به منظور تامین و بهبود سلامت روان جامعه ارائه می‌دهد.
		
		\subsection{ضرورت انجام پروژه}
		امروزه فضاهای مختلفی هستند که خدمات مختلف مربوط به روان، همچون نوبت دهی مشاوره و روانپزشک، آموزش‌های جامع، نسخه دهی آنلاین، داروخانه غیرحضوری و تست های تخصصی روانشناسی را ارائه می‌کنند.
		به منظور ارائه یکپارچه این خدمات ما تصمیم به ایجاد پلتفرم روان سبز گرفتیم تا همراه با ایجاد اشتغال، سلامت روان جامعه را بهبود دهیم و همچنین دانش افراد را در رابطه با روان افزایش دهیم و نیز به سود اقتصادی قابل توجهی برسیم. هدف ما ارائه بهترین آموزش ها، باکیفیت ترین خدمات درمان و نیز تحویل دارو با سرعت و به آسان ترین حالت و با بهیاری حرفه ای می‌باشد.
		از دیگر مزیتهای پلتفرم ما از بین بردن محدودیت زمانی و مکانی و امکان ارتباط با بهیاری در همه ساعات شبانه روز می‌باشد.
		
		\subsection{جنبه‌های نوآوری}
				تولید پادکست‌های هفتگی با حضور موفق‌ترین روانپزشکان و روانشناسان یکی از اصلی ترین نوآوری‌های سیستم است و به کاربران اجازه می‌دهد حتی بعد از مدت درمان و مشاوره گرفتن همچنان با ما از طریق پادکست باشند.
				دیگر کار مهمی که به عنوان فرایندی نو انجام می‌گردد، تسهیل نسخه دهی و دارو رسانی با اتصال به داروخانه‌ها است که بسیاری از مشکلات بیماران برای یافتن یک داروخانه معتبر و در ارتباط با مجموعه روان سبز را حل می‌کند.
				پیگیری مراحل درمان بیماران، یکی دیگر از دغدغه‌های پلتفرم است که از طریق بهیار‌های پلتفرم صورت می‌پذیرد.
				
		
		\subsection{خدمات قابل ارائه}
پلتفرم روان سبز با تاکید بر دغدغه اصلی خود یعنی بهبود سلامت روان جامعه، تلاش دارد تا با ارائه خدماتی که در ادامه ذکر میشود گامی بلند جهت تحقق رسالت خود بردارد.
		\\
		
		\begin{enumerate}
			\item مرکز تخصصی مشاوره
			\\
			این مرکز به عنوان یکی از بزرگترین بخش های مجموعه تلاش دارد تا پاسخگوی انواع بخش های مشاوره اعم از:
			\begin{itemize}
				\item مشاوره تحصیلی
				\item مشاوره شغلی
				\item مشاوره پیش از ازدواج
				\item مشاوره سازشی یا شخصی
				\item مشاوره اجتماعی و مشاوره های تخصصی نظیر مشاورۀ حقوقی و پزشکی
				\item مشاوره و استعدادیابی کودک و نوجوان
			\end{itemize}
			
			\item مرکز روان پزشکی
			\\
نرم افزار ما همچنین با فراهم سازی روانپزشکان با تجربه و متخصص در زمینه های گوناگون تلاش دارد در عرصه درمان نیز قدم بگذارد.

			\item نسخه دهی و ارتباط با داروخانه ها
			\\
			وقتی صحبت از پزشک و درمان می شود دارو و تسهیل فرایند آن یکی از مهم ترین مسائل است. نرم افزار ما جهت تسهیل این روند با داروخانه های طرف قرارداد و بیمه های صاحب نام در ارتباط خواهد بود تا فرایند نسخه دهی الکترونیک نیز به بهترین نحو صورت پذیرد.

			\item امتیاز دهی و ثبت نظرات درباره پزشکان
			\\
			هر پزشک که توسط نرم افزار تایید صلاحیت می شود دارای پروفایل اختصاصی خود است که شامل شماره نظام وظیفه پزشک می‌باشد. طبیعتا با هر مراجعه میزان رضایت فرد مورد نظر نیز باید مورد ارزیابی قرار گیرد تا مراجعین بتوانند از نظرات یکدیگر جهت تصمیم گیری بهتر استفاده کنند. همچنین افراد میتوانند در صفحات شخصی پزشکان نظرات خود را ثبت کنند.
			\item اطلاعات جامع پزشکان
			\\
			علاوه بر امتیازدهی مراجعی می توانند با مشاهده متخصصین و استفاده از قابلیت فیلتر در سایت، با توجه به رزومه و نیز نیاز شخصی، بهترین انتخاب را داشته باشند. اطلاعات جامع پزشکان همچنین شامل ویدیو معرفی و میزان رضایت مراجعین می‌باشد.
			
			\item انجام تست های روانشناسی معتبر
			\\
			حتی در صورت علاقه نداشتن به انجام مشاوره، مراجعین می توانند از خدمات دیگر نرم افزار مانند تست های روان شناختی معتبر استفاده نمایند. این تست ها میتوانند به شناخت هر چه بیشتر پزشکان از بیمار نیز کمک کند.
			
			\item حفظ امنیت و حریم بیماران
			\\
			نرم افزار و تیم مدیریت مجموعه در قبال هرگونه درز اطلاعات خصوصی بیمار مسئول است و راهکارهای بازدارنده آن را مانند تعهد گیری از پزشکان و افزایش امنیت پایگاه داده و سایت انجام خواهد داد.
			
			\item نوبت دهی انعطاف پذیر با توجه به خواست بیمار
			\\
			بسته به انتخاب هر شخص، امکان مراجعه به پزشک و مشاور به طرق مختلف وجود دارد. این راه ها میتواند غیرحضوری مانند ارتباط تلفنی (صوتی)، ویدیویی ، متنی (به صورت چت در نرم افزار) باشد و یا حضوری باشد. نرم افزار محدودیتی از لحاظ پذیرش پزشکان از شهرهای مختلف ندارد و مراجعین برای مراجعه حضوری میتوانند با فیلتر کردن شهر خود نتایج بهتری بگیرند.
			
			\item ارائه مجله های هفتگی
			\\
			نرم افزار ما علاوه بر بیماران، برای علاقه مندان به روانشناسی نیز می‌تواند مفید باشد. به همین منظور مجله هفتگی با محتوای جذاب حاوی به روز ترین تحقیقات دانشمندان در حوزه روانشناسی و دستاوردهای مهم این علم، ارائه می‌شود.
			
			
		\end{enumerate}
				
		
		\subsection{استفاده کنندگان اصلی از نرم‌افزار (نقش‌ها)}
		نرم افزار شامل بخش های متنوع و در نتیجه طیف های مختلفی از مصرف کنندگان است. به طور کل کسانی که از مشکلات روانی رنج می برند اصلی ترین نقش را در بین استفاده کنندگان از سیستم ایفا می کنند. علاقه مندان به مسائل روانشناسی نیز میتوانند از امکانات مجله هفتگی نرم افزار استفاده کنند.
\\	همچنین پزشکان نیز به عنوان بخش مهمی از استفاده کنندگان محسوب می شوند.
\\	بخش دیگری از نقش ها، بهیار های سیستم هستند که به صورت های مداوم باید مشغول باشند.
\\   همچنین یک نقش دیگر، داروخانه‌ها هستند که نتیجه استعلام را اعلام می‌کنند.
		
		
		\subsection{سیستم‌های مشابه و سرویس‌های ارائه شده}
		چندین سرویس مانند سرویسی که مد نظر ما هست ارائه شده است. هر کدام از این سیستم‌ها مزایا و معایبی دارند. در ادامه به بررسی دو نمونه از این سرویسها میپردازیم:
		\begin{enumerate}
			
		\item \href{https://www.drsaina.com/consultation?gclid=CjwKCAjw7c2pBhAZEiwA88pOF8SCqb0W9MbWxGtNfhIICKenmpTzlyl0pyB7eLBUyRKcnRM4W1w9wRoClzsQAvD_BwE}{دکتر ساینا}\\
		وب سایت دکتر ساینا خدماتی مشابه با سیستم ما ارائه میدهد. با بررسی و دیدن جزییات این سامانه میتوانیم نکاتی را بررسی نماییم. 
		\begin{itemize}
			\item وب سایت دکتر ساینا با سرچ در گوگل جزو اولین وب سایت‌هایی خواهد بود که نمایش داده میشود از این رو میتوان گفت این سایت از سئو خوبی برخوردار است. این سایت توانسته برای مراجعین محیط درمانی جامعی را بسازد. همچنین این سایت رابط کاربری زیبا و کاربر پسندی دارد.
			
			\item از معایب این سایت میتوان به محدودیت در سرچ و بهیاری ضعیف آن اشاره کرد زیرا تنها می‌توان از جملات محدودی استفاده کرد.
\\		با بررسی بیشتر وب سایت متوجه غیر منطقی بودن اطلاعات و آمار ها شدیم که سبب می‌شود اعتبار مجموعه زیر سوال برود و بیماران نتوانند به راحتی اعتماد کنند.
\\		از دیگر نقوص، زمان محدود مشاوره آنلاین و یا تلفنی است که تنها 20 دقیقه می باشد که غیرمنطقی است و نیاز مراجعین پاسخ داده نمی‌شود.
\\		از دیگر مشکلات دیده شده می‌توان به نبودن بهیاری درست اشاره کرد برای مثال کامنت های کاربران به هیچ عنوان پاسخ داده نمی شوند.
\\		کامل نبودن اطلاعات پزشکان، مشخص نبودن بیمه های طرف قرارداد وهزینه غیرمنطقی جلسات از دیگر نکات منفی این سامانه می‌باشد.
		\end{itemize}
		
		\item \href{https://www.google.com/url?sa=t&rct=j&q=&esrc=s&source=web&cd=&cad=rja&uact=8&ved=2ahUKEwjj8qObiomCAxVxVKQEHShlDS0QFnoECA0QAQ&url=https%3A%2F%2Fwww.darmankade.com%2F&usg=AOvVaw3kmECL-fyw2GEWMgRr5Eir&opi=89978449}{درمانکده}\\
		سایت دیگری که به بررسی آن پرداختیم سایت درمانکده میباشد.
		\begin{itemize}
			\item از ویژگی های خوب این وب‌سایت می‌توان به وجود پروفایل مجزا برای هر کاربر، قرار دادن تلفن ثابت برای جلب اعتماد مشتری و ساعات بهیاری بالا اشاره کرد.
			\item در این وب‌سایت اطلاعات کافی درباره پزشکان وجود ندارد و تخصص آن‌ها به طور دقیق مشخص نیست.همچنین در این سایت برای مشاوره حضوری فقط شهر تهران قرار داده شده و در دیگر شهرها امکان مشاوره حضوری وجود ندارد.
		\end{itemize}
		
		
	\end{enumerate}
		
		\subsection{توجیه اقتصادی پروژه}
		\subsubsection{تحلیل هزینه}
		طبق تحلیل انجام شده برای انجام این پروژه به 2 مهندس کامپیوتر با تجربه متوسط و 1 مهندس کامپیوتر با تجربه سنیور نیاز داریم.درآمد پیشبینی شده آنها حدود 40 میلیون تومان برای مهندسان با تجربه متوسط و حدود 60 میلیون تومان برای مهندس سنیور میباشد. این درآمد مربوط به طراحی اولیه و کلی سایت است.و در ادامه ماهیانه حدود 3 میلیون تومان برای یکی از مهندسین به عنوان بهیار در نظر گرفته شده است.
		برای تبلیغات هزینه ای حدود 25 میلیون تومان در نظر گرفته شده است.
		برای تیم بهیاری به 2 نفر و برای هر کدام ماهیانه 8 میلیون تومان در نظر گرفتیم.
		برای فیلم‌بردار و عکاس به یک فرد با تجربه متوسط و حقوق ماهیانه 8 میلیون تومان نیاز داریم.
		
		\subsubsection{تخمین درآمد}
		برای داروخانه 18 درصد پورسانت در نظر گرفتیم.
		همچنین دوره های آموزشی که در طول ماه در وب سایت برگزار میکنیم هم درآمد خوبی برایمان خواهند داشت.
		طبق بررسی های انجام شده در این راه حدود 6 ماه توسعه محصول و جذب نیرو داریم و همچنین انشالله بعد از 9 ماه به سوددهی می‌رسیم. همچنین طبق نتایج و سودهای بدست امده میتوانیم نیروهای کاری را هم بیشتر کنیم و استخدام گسترده ای را شروع کنیم.
				
		\subsection{افراد پروژه}
		تیم نرم افزاری ما متشکل از سه نفر است که مهارت های نرم افزاری لازم مانند دانش و تجربه ی کار در زمینه ی فرانت اند، بک اند، دیتابیس و مستندسازی، روحیه كار تيمي بالا، توانایی یادگیری، مديريت و برنامه ريزى را دارا مى‌باشند.
		
		\subsection{تاثیر مثبت در آینده کاری}
		از جمله تاثیرات مثبت پروژه می توان به ایجاد روحیه کمک رسانی در خود و افزایش تجربه در زندگی اجتماعی نام برد.
\\	ایجاد یک رزومه کاری و استفاده از درآمد ناشی از امور مختلف در پروژه برای ایجاد کسب و کارهای مورد علاقه دیگر
		
		\subsection{ایده برای آینده برنامه}
		
		\begin{itemize}
			\item با انجام پروژه و پیشرفت مورد نیاز میتوانیم زمینه سازی برای ایجاد یک شبکه بزرگ از روانشناسان و روانپزشکان با مجموعه جهانی را آغاز کنیم تا بتوانیم از دانش حداکثری برای رفاه حال عموم استفاده کنیم.
			
			\item تولید \lr{Application} موبایل برای سیستم عامل های \lr{ios} و \lr{Android} جهت سهولت دسترسی به پلتفرم
				
			\item ایجاد محیطی برای تمرین‌های تاثیرگذار روانشناسی
			
			\item ایجاد پیام رسان داخلی پلتفرم برای ارتباط مستقیم با پزشکان بدون رزرو قبلی (چت کردن) و بدون هزینه
			
			\item تولید محتوا و دوره های آموزشی برای اقشار مختلف مانند والدین برای تربیت فرزندان و هدایت تحصیلی فرزندان‌شان
			
			\item ایجاد شبکه رادیویی مجزا با حضور روانشناسان و پزشکان و پرسش و پاسخ
			
		\end{itemize}
		
		\section{بخش دوم: تاثیرگذاری}
		\subsection{قابلیت چاپ در قالب مقاله و يا گزارش علمي}
		هدف ما نوآوری در رابطه با خدمت ارائه شده می‌باشد و قابلیت نرم افزاری جدیدی را پوشش نمی‌دهیم که امکان چاپ مقاله ایجاد شود.
		
		\subsection{اطلاع رسانی نتايج كار به گروه هاي هدف، شامل كاربران و بقيه افراد درگير}
		بخش قابل توجهی از سرمایه خود و نیز درامد بعد از رسیدن به سود را به تبلیغات گسترده و با هدف، اختصاص می‌دهیم.
		
		\subsection{تبادل دو طرفه با مشتری}
		بخش پیشنهادات و انتقادات برای نظر دادن درباره همه خدمات ما وجود دارد. هم چنین تیم بهیاری در همه ساعات شبانه روز آماده به خدمت رسانی هستند. در ارتباط با خدمات درمانی بعد از هر نوبت با پزشک، فرد میتواند نظر خود را درباره جلسه و درمان ارائه شده به صورت عمومی منتشر کند و تیم بهیاری در صورت نیاز به پزشک ارائه می‌دهد.
		
		\section{بخش سوم: پیاده سازی}
		\subsection{تقسیم فازها}
		طبق برآوردی که داشتیم فازهای پروژه را به صورت زیر تقسیم کرده ایم:
		\begin{itemize}
			\item فاز 1: جذب سرمایه
			\item فاز 2: جذب نیرو
			\item فاز 3: ایجاد زیرساخت سایت
			\item فاز 4: ثبت نیرو و درخواست مجوز برای دسترسی های قانونی
			\item فاز 5: پیاده سازی طراحی \lr{ui}
			\item فاز 6: بررسی و تایید طراحی
			\item فاز 7: نوشتن \lr{api} های مورد نیاز و بک اند لازم برای وب سایت
			\item فاز 8: طراحی فرانت وب سایت و استفاده \lr{api} ها
			\item فاز 9: تست وب سایت توسط تسترها و اعلام باگهای موجود
			\item فاز 10: رفع باگهای رسیده و انتشار ریلیز اول
		\end{itemize}
		
		\subsection{بیان برنامه احتمالی}
		\begin{itemize}
			\item جذب سرمایه حدود ۲ ماه
			\item جذب نیرو حدود 8 هفته
			\item ایجاد زیرساخت سایت 5 هفته
			\item فاز 4 حدود ۴ هفته
			\item فاز 5: حدود 1.5 ماه
			\item فاز 6:   2 روز
			\item فاز 7: حدود 2 هفته
			\item فاز 8: حدود 1.5 ماه
			\item فاز 9: حدود 5 روز
			\item فاز 10: حدود 5 روز
		\end{itemize}
		\subsection{بررسی ریسک‌های احتمالی پروژه}
		
		\begin{itemize}
			\item مشکلات امنیتی: این شامل درز اطلاعات خصوصی بیماران، دسترسی غیرمجاز به پایگاه داده‌ها و سایت، ویروس‌ها و بدافزارها و سایر مشکلات امنیتی می‌شود.
			\item مشکلات فنی: این شامل خطاهای نرم‌افزاری، اختلالات در سرورها و شبکه‌ها، مشکلات در نسخه دهی و سایر مشکلات فنی می‌شود.
			\item مشکلات مالی: این شامل مشکلات مربوط به تامین مالی پروژه، ناپایداری در قیمت‌گذاری، مشکلات مربوط به پرداخت و دریافت وجه می‌شود.
			\item مشکلات مربوط به قوانین و مقررات: این شامل مشکلات مربوط به رعایت قوانین حریم خصوصی، قوانین پزشکی و سایر قوانین و مقررات مربوط به حوزه پزشکی و روانشناسی می‌شود.
		\end{itemize}
	
	

\end{document}
