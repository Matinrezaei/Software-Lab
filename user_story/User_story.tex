\documentclass[a4paper]{article}
\input{etc/cmd}
\settextfont[Scale=1.2,BoldFont=B Titr]{B Zar}

\begin{document}

	{ \centering
		\begin{figure}[ht]
			\centering
			\includegraphics[width=0.5\linewidth]{etc/IUTLogo}
			\label{fig:fig1}
		\end{figure}
		
		\fontsize{28pt}{35pt}{\textbf{دانشگاه صنعتی اصفهان}}\\
		
		\vspace{1cm}
		
		\fontsize{28pt}{35pt}{\bf دانشکده برق و کامپیوتر}
		
		\vspace{1cm}
		
		\fontsize{25pt}{35pt}{\bf یوزر استوری‌های پروژه مهندسی نرم‌افزار}
		
		\vspace{2cm}
		
	 \begin{huge}
	 	\textbf{اعضای گروه} \\
	 	\fontsize{20}{20} \vspace{0.5cm}
	 	\begin{center}
	 		پردیس یاوری -	پویا غنی - متین رضایی
	 	\end{center}
	 
	 \end{huge}
}
	
	\newpage
	\thispagestyle{empty}
	\pagenumbering{arabic}
	\tableofcontents
	
	\newpage

	\fontsize{12pt}{14pt}\selectfont
	
	\pagestyle{fancy}

	\setlength\headheight{26pt}
	
	\rhead{\begin{picture}(3,3) \put(3,3){\includegraphics[width=2cm]{etc/IUTLogo}} \end{picture}}
	
	\lhead{\begin{subject}
			یوزر استوری های پروژه مهندسی نرم‌افزار
			\vspace{1ex plus 0.5ex minus 0.5ex}
	\end{subject}
	}
	
	\section{\lr{User Stories}}
	\subsection{کاربر بدون نقش}
با وارد کردن \lr{URL} سایت، وارد صفحه اصلی پلتفرم روان سبز می‌شویم.
در این صفحه در منوی کاربر گزینه مجله هفتگی، گزینه پادکست‌ها، انجام تست های روان شناسی و گزینه دیدن پزشکان و روان‌شناسان را می‌بیند.
همچنین با فشردن دکمه ورود صفحه می‌تواند با وارد کردن نام کاربری و رمزعبور وارد پنل کاربری مخصوص خود شوند.
اگر کاربر تاکنون ثبت نام نکرده باشد می‌تواند در صفحه ورود گزینه ثبت نام را انتخاب کند.
انجام عملیات ثبت نام برای همه نقش ها می‌باشد. بیماران، بهیاران و پزشکان می‌توانند با وارد کردن مقادیری که برای‌شان مشخص شده است، این عملیات را انجام دهند.

	\begin{itemize}
		\item با انتخاب گزینه مجله هفتگی، به طور پیش فرض آخرین مجله منتشر شده پلتفرم ما نمایش داده می‌شود. همچین در انتهای این صفحه به طور لیست امکان باز کردن مجله های گذشته را نیز خواهیم داشت.
		
		\item همچنین در این صفحه می‌توانیم با انتخاب گزینه مشارکت با استفاده از یک \lr{popup}، متن یا فایلی که میخواهیم برای چاپ شدن در مجله پیشنهاد کنیم را همراه با نام و نام خانوادگی و ایمیل قرار دهیم.\\
		این پیشنهادها توسط ادمین روان سبز بررسی و در صورت انتخاب شدن با نام شخص در مجله قرار داده می‌شود.
		
		\item دیگر انتخابی که می‌توانیم حتی به عنوان کاربر مهمان داشته باشیم، پادکست‌های تولید شده روان سبز است که عموم می‌توانند از آن استفاده کنند و در صفحه باز شده برای پادکست ها امکان نظر دادن و مشاهده نظرات دیگر کاربران را خواهیم داشت.
		
		\item با انتخاب گزینه های پزشکان و روان‌شناسان نیز می‌توانیم لیستی از آنها را مشاهده کنیم که مطالب تکمیلی تر در مورد این صفحات در بخش بیمار بیشتر توضیح داده شده است.
		تفاوتی که در این بخش بین کاربر مهمان و بیمار وجود دارد امکان نوبت گیری است که کاربر مهمان این امکان را نخواهد داشت.
		
		\item در بخش تست ها با انتخاب این گزینه ما تمام تست های موجود را مشاهده می‌کنیم و می‌توانیم آنها را انتخاب کنیم.\\
		با انتخاب هر یک از آنها تست ها به صورت تک تک می‌آید و گزینه خود را انتخاب می‌کنیم. بعد از اتمام تست، نتیجه به ما نشان داده خواهد شد.
		اگر ورود کرده باشیم در دفعات بعد از طریق گزینه سوابق تست ها که در همین صفحه است می‌توانیم نتایج تست های قبل خود را ببینیم و در صورت نیاز نتایجش را با پزشک خود در میان بگذاریم.
	\end{itemize}

	\subsection{پزشک}

پزشک نیز با انتخاب گزینه ثبت‌نام باید فرم مختص خود شامل نام و نام خانوادگی، کدملی، شماره تلفن، ایمیل (اختیاری)، نام کاربری و رمز‌عبور را کامل کرده و ثبت‌نام خود را تکمیل کند.\\
سپس به مانند بقیه نقش‌ها، با وارد کرد نام کاربری و رمز عبور وارد پنل مخصوص پزشکان می‌شود.
پزشک با وارد شدن به پنل کاربری خود با گزینه های زیر مواجه می‌شود:
		
		\subsubsection{لیست بیماران ویزیت شده}
		در قسمت لیست بیماران ویزیت شده پزشک می‌تواند اسامی‌و بیماران ویزیت شده قبلی اش و همچنین نسخه هایی که برای آنها نوشته را ببیند.
		با انتخاب هر بیمار می‌تواند وضعیت نوشته شده توسط بهیار را مطالعه کند. همچنین اطلاعات تماس بیمار را می‌تواند مشاهده کند.
		\subsubsection{لیست نوبت‌ها}
		در قسمت لیست نوبت‌ها پزشک می‌تواند لیست نوبت‌هایی که بیماران با او گرفته اند را به همراه مشخصات بیمار و زمان نوبت آنها و همچنین حضوری یا غیرحضوری بودنشان را ببیند.
		\subsubsection{جلسات غیرحضوری}
		در بخش ورود به اتاق معالجه پزشک وارد سامانه ای می‌شود که همزمان بیماری که در آن زمان نوبت دارد هم وارد می‌شود می‌توانند جلسه درمان را پیش ببرند. همانطور که در بخش بیمار نیز در این مورد صحبت شده بعد از گرفتن نوبت غیرحضوری، در مکان نوبت‌ها لینک جلسه برای بیمار گذاشته خواهد شد.
		\subsubsection{نوشتن نسخه}
		همچنین در بخش نوشتن نسخه پزشک می‌تواند با دیدن نام بیماری که معالجه کرده برای او نسخه اش را تنظیم کند و برایش ارسال کند.
		\subsubsection{مشاهده بهیاران}
		با انتخاب بخش بهیار‌ها، پزشک وارد صفحه‌ای می‌شود که می‌تواند تمام بهیار‌های ثبت‌نام شده در پلتفرم را مشاهده کند. پزشک می‌تواند با انتخاب هر بهیار اطلاعات و سابقه وی را مورد بررسی قرار داده و در صورت پذیرفتن آن، با انتخاب گزینه ثبت درخواست، درخواستش برای بهیار مورد نظر ارسال می‌شود.\\
		\subsubsection{احراز هویت}
		در بخش احراز هویت پزشک، پزشک باید گواهی نظام‌پزشکی خود را آپلود و تعهدی که خود پلتفرم روان سبز از پزشک بابت حریم خصوصی بیمار می‌گیرد را تکمیل کند.
		
	\subsection{بهیار}

برای ورود به عنوان این نقش همانند بقیه نقش ها به نام کاربری و رمز عبور احتیاج داریم.
این افراد بهیار، ثبت نام از طریق سامانه را به مانند پزشکان و بیماران با همان فرم خواهند داشت. این افراد می‌توانند اطلاعات‌شان را در سامانه تکمیل کنند (احراز).\\
در پروفایل عمومی‌خود می‌توانند سابقه و رزومه خود را منتشر کنند تا پزشکان بر اساس آن درخواست‌هایشان را به بهیار‌ها ارسال کنند.
بهیار هم در صفحه اصلی خود بخش وضعیت احراز هویت خود را دارد که در آن به طور کل پروفایل خود را تکمیل می‌کند.
بعد از وارد شدن به عنوان بهیار، این موارد را مشاهده خواهد کرد:
		
		\begin{itemize}
			\item مجله هفتگی
			\item پادکست ها
			\item درخواست ها
			\item پیگیری ها
		\end{itemize}

مجله هفتگی، پادکست و تست ها مطابق آنچه در بخش کاربر بدون نقش گفته شد انجام میگردد.

در بخش درخواست‌ها ما در واقع به درخواست‌هایی که از طریق پزشکان برای ما ارسال شده پاسخ می‌دهیم. روبروی هر درخواست دو گزینه حذف و گزینه قبول را میبینیم که با انتخاب قبول به عنوان بهیار پزشک انتخاب و به بیماران ایشان دسترسی خواهیم داشت.

در بخش پیگیری‌ها ما بیمارانی که با بهیار پزشکان آنها هستیم را مشاهده می‌کنیم.
روبروی هر بیمار نام پزشک آن نوشته شده است.
طبق زمانبندی و دوره‌ای که خود پزشک برای بیمار تعریف کرده ما یک سری تسک را می‌بینیم.
بعد از تماس و انجام کارهای لازم ما آن مورد را به حالت \lr{checked} در می‌آوریم.

	\subsection{بیمار}
بعد از وارد شدن به بخش ورود بیمار، کاربر می‌تواند با وارد کردن نام کاربری و پسورد خود وارد پنل خود شود.
همچنین در این صفحه می‌تواند در صورتی که هنوز ثبت نام نکرده روی گزینه ثبت نام بزند. بعد از وارد شدن به بخش ثبت نام باید فرم مخصوص بیمار را تکمیل کند. این فرم شامل نام و نام خانوادگی، شماره تماس، ایمیل (اختیاری)، کدملی، نام کاربری، رمز عبور می‌باشد. در نهایت پس از تکمیل وارد مرحله تایید شامل چک کردن یکتا بودن نام کاربری و وارد کردن پیامک تاییدیه ارسال شده به تلفن همراه ثبت نام کامل می‌شود.
\\
گزینه هایی که بیمار بعد از وارد شدن در پنل خود میبیند را در ادامه مشاهده و به ترتیب با هر روند هر گزینه آشنا میشویم:

		\subsubsection{پزشکان و روان‌شناسان}
		در این ابتدا انتخاب می‌کنیم که به دنبال پزشک می‌گردیم یا روان‌شناس. پس از انتخاب یکی از آنها می‌توانیم از بخش فیلتر پزشکان براساس تخصص (تخصص های مشخص)، نظرات کاربران، نوع مراجعه، شهر و بخش سرچ کردن اسامی‌است، استفاده کنیم.\\
		بعد از مشخص کردن پزشک یا روان‌شناسان تخصص‌ها متناسب با گزینه انتخابی به روز می‌گردد.\\
			در این صفحه ما لیستی از پزشکان را خواهیم داشت و می‌توانیم با انتخاب اسم آنها وارد پروفایل اختصاصی آنها شویم.
			هم در صفحه پزشکان و هم در پروفایل هر پزشک می‌توانیم با انتخاب گزینه دریافت نوبت، نوبت خود را رزرو کنیم. اگر نوبت غیرحضوری از طریق صوتی یا تصویری را انتخاب کنیم، لینک اتاق مجازی پزشک مورد نظر نمایش داده می‌شود.
			
			با انتخاب گزینه انتخاب نوبت یک صفحه \lr{popup} برای ما باز خواهد شد که می‌توانیم بر حسب روز نوبت های خالی موجود (تعدادی ساعت مشخص) را مشاهده و یکی را انتخاب کنیم. (همانطور که در نیازمندی ها گفته شد از هر پزشک یک نوبت تا موعد مقرر بگیرد)
			پرداخت نیز به صورت آنلاین انجام می‌شود.
			
		\subsubsection{دریافت نسخه}
			در بخش دریافت نسخه ها همانطور که گفته شد می‌توانیم ابتدا نسخه های خود را مشاهده بکنیم.
			در هر نسخه نمایش داده شده، دکمه استعلام موجود است. پس از انتخاب این گزینه یک \lr{popup} نمایش داده می‌شود که لیستی از داروخانه هاست که می‌توان بر اساس شهر فیلتر نمود. با انتخاب هر دارو خانه، اقلام موجود در نسخه را از آن داروخانه استعلام میگیرد.
			بعد از پاسخ دهی داروخانه در همان صفحه نسخه ها می‌توان وضعیت را مشاهده کنیم که کدام‌یک از اقلام موجود در نسخه را داروخانه می‌تواند تأمین کند و کدام‌یک را نمی‌تواند.
		\subsubsection{تاریخچه نسخه ها}
			در این قسمت می‌تواند نسخه‌های جاری و پیشین خود را ببیند. این کار به مانند دفترچه بیمه عمل می‌کند.
		\subsubsection{تاریخچه نوبت ها}
			در بخش تاریخچه نوبت‌ها می‌توانیم نوبت‌های گرفته شده خود را مشاهده کنیم. در کنار هر نوبت گزینه لغو نوبت را مشاهده می‌کنیم. با انتخاب این گزینه، اگر حداقل یک روز به نوبت مانده باشد، نوبت با موفقیت لغو می‌گردد و تمامی‌وجه بازگشت داده می‌شود. در غیر این صورت نیمی ‌از هزینه بازگشت داده می‌شود.
		\subsubsection{تست های روانشناسی}
		دقیقا مانند چیزی که در کاربر بدون نقش نوشته شده است.
		\subsubsection{مجله هفتگی}
		دقیقا مانند چیزی که در کاربر بدون نقش نوشته شده است.

		\subsubsection{احراز هویت}
همچنین هر کاربر در بالای پنل کاربری خود می‌تواند وضعیت احراز هویت خود را مشاهده کند. اگر هنوز احراز نشده باشد گزینه احراز هویت وجود دارد که با انتخاب آن وارد بخش احراز هویت می‌شود.
بیمار برای دریافت نسخه های خود حتما باید احراز هویت شده باشد. صفحه احراز هویت شامل بخش آپلود رو و پشت کارت ملی و انتخاب بیمه می‌باشد.

	\subsection{داروخانه}
		داروخانه پس از ثبت‌نام شدن توسط ادمین می‌تواند با وارد شدن به بخش پاسخ دهی به نسخ، نسخه‌هایی را که بیماران برای استعلام به آنها اعلام کرده‌اند را مشاهده کند و برای هر دارو بازخوردش را ثبت کند.

\end{document}
